\section{Meet-in-the-middle}

Meet-in-the-middle разбивает задачу пополам и решает всю задачу через частичный расчет половинок. Он работает следующим образом: переберем все возможные значения x и запишем пару значений $(x,f(x))$ в множество. Затем будем перебирать всевозможные значения $y$, для каждого из них будем вычислять $g(y)$, которое мы будем искать в нашем множестве. Если в качестве множества использовать отсортированный массив, а в качестве функции поиска — бинарный поиск, то время работы нашего алгоритма составляет $O(XlogX)$ на сортировку, и $O(YlogX)$ на двоичный поиск, что дает в сумме $O((X+Y)logX)$.

\subsection{Задача о нахождении четырех чисел с суммой равной нулю}

Дан массив целых чисел $A$. Требуется найти любые 4 числа, сумма которых равна 0 (одинаковые элементы могут быть использованы несколько раз).

Например: $A=(2,3,1,0,-4,-1)$. Решением данной задачи является, например, четверка чисел $3+1+0-4=0$ или $0+0+0+0=0$.

Наивный алгоритм заключается в переборе всевозможных комбинаций чисел. Это решение работает за $O(N^4)$. Теперь, с помощью Meet-in-the-middle мы можем сократить время работы до $O(N^{2}logN)$.

Для этого заметим, что сумму $a+b+c+d=0$ можно записать как $a+b=-(c+d)$. Мы будем хранить все $N^2$ пар сумм $a+b$ в массиве $sum$, который мы отсортируем. Далее перебираем все $N^2$ пар сумм $c+d$ и проверяем бинарным поиском, есть ли сумма $-(c+d)$ в массиве $sum$. Кроме этого можно отсортировать оба списка $a+b$ и $-(c+d)$, а замет двигаться по ним параллельно.

Итоговое время работы $O(N^{2}logN)$.

Если вместо отсортированного массива использовать хэш-таблицу, то задачу можно будет решить за время $O(N^2)$.

\subsection{Задача о рюкзаке}

Классической задачей является задача о наиболее эффективной упаковке рюкзака. Каждый предмет характеризуется весом ($w_{i} \le 10^{9}$) и ценностью ($cost_{i} \le 10^{9}$). В рюкзак, ограниченный по весу, необходимо набрать вещей с максимальной суммарной стоимостью. Для ее решения изначальное множество вещей $N$ разбивается на два равных(или примерно равных) подмножества, для которых за приемлемое время можно перебрать все варианты и подсчитать суммарный вес и стоимость, а затем для каждого из них найти группу вещей из первого подмножества с максимальной стоимостью, укладывающуюся в ограничение по весу рюкзака. Сложность алгоритма $O({2^{\frac{N}{2}}}\times{N})$. Память $O({2^{\frac{N}{2}}})$.

Разделим наше множество на две части. Подсчитаем все подмножества из первой части и будем хранить их в массиве $first$. Отсортируем массив $first$ по весу. Далее пройдемся по этому массиву и оставим только те подмножества, для которых не существует другого подмножества с меньшим весом и большей стоимостью. Очевидно, что подмножества, для которых существует другое, более легкое и одновременно более ценное подмножество, можно удалять. Таким образом в массиве $first$ мы имеем подмножества, отсортированные не только по весу, но и по стоимости. Тогда начнем перебирать все возможные комбинации вещей из второй половины и находить бинарным поиском удовлетворяющие нам подмножества из первой половины, хранящиеся в массиве $first$.

Итоговое время работы $O({2^{\frac{N}{2}}}\times({N}+\log{2^{\frac{N}{2}}})) = O({2^{\frac{N}{2}}}\times{N})$.

\subsection{Задача о нахождении кратчайшего расстояния между двумя вершинами в графе}

Еще одна задача, решаемая Meet-in-the-middle — это нахождение кратчайшего расстояния между двумя вершинами, зная начальное состояние, конечное состояние и то, что длина оптимального пути не превышает $N$. Стандартным подходом для решения данной задачи, является применение алгоритма обхода в ширину. Пусть из каждого состояния у нас есть $K$ переходов, тогда бы мы сгенерировали $K^N$ состояний. Асимптотика данного решения составила бы $O(K^N)$. Meet-in-the-middle помогает снизить асимптотику до $O(K^{\frac{N}{2}})$.

Алгоритм решения:
\begin{itemize}
    \item Сгенерируем BFS-ом все состояния, доступные из начала и конца за $\frac{N}{2}$ или меньше ходов.
    \item Найдем состояния, которые достижимы из начала и из конца.
    \item Найдем среди них наилучшее по сумме длин путей.
\end{itemize}

Таким образом, BFS-ом из двух концов, мы сгенерируем максимум $O(K^{\frac{N}{2}})$ состояний.

\subsection{Дискретное логарифмирование}
\textbf{Необходимо добавить материал! Алгоритм Гельфонда—Шенкса}
% TODO https://ru.wikipedia.org/wiki/Алгоритм_Гельфонда_—_Шенкса
% http://www.e-maxx-ru.1gb.ru/algo/discrete_log


\subsection{Задания по теме}
\subsubsection{Теоретическое задание}
Предложите алгоритм, который по заданному числу $N$ находит количество четверок целых положительных чисел $(x, y, z, t)$ - решений уравнения: $$x^{2} + y^{2} + z^{2} + t^{2} = N.$$

Время работы алгоритма: $O(N)$ или $O(NlogN)$.

\subsubsection{практическое задание}
Дано $n (n \le 40)$ предметов, каждый из них характеризуется весом и ценностью.

Необходимо собрать рюкзак весом не больше $W$ такой, что суммарная ценность всех вещей в рюкзаке будет максимально возможной. Другими словами, необходимо выбрать подмножество предметов с ограничением на суммарную массу с максимальной ценностью.
